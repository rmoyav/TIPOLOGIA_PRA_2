% Options for packages loaded elsewhere
\PassOptionsToPackage{unicode}{hyperref}
\PassOptionsToPackage{hyphens}{url}
%
\documentclass[
]{article}
\usepackage{amsmath,amssymb}
\usepackage{lmodern}
\usepackage{ifxetex,ifluatex}
\ifnum 0\ifxetex 1\fi\ifluatex 1\fi=0 % if pdftex
  \usepackage[T1]{fontenc}
  \usepackage[utf8]{inputenc}
  \usepackage{textcomp} % provide euro and other symbols
\else % if luatex or xetex
  \usepackage{unicode-math}
  \defaultfontfeatures{Scale=MatchLowercase}
  \defaultfontfeatures[\rmfamily]{Ligatures=TeX,Scale=1}
\fi
% Use upquote if available, for straight quotes in verbatim environments
\IfFileExists{upquote.sty}{\usepackage{upquote}}{}
\IfFileExists{microtype.sty}{% use microtype if available
  \usepackage[]{microtype}
  \UseMicrotypeSet[protrusion]{basicmath} % disable protrusion for tt fonts
}{}
\makeatletter
\@ifundefined{KOMAClassName}{% if non-KOMA class
  \IfFileExists{parskip.sty}{%
    \usepackage{parskip}
  }{% else
    \setlength{\parindent}{0pt}
    \setlength{\parskip}{6pt plus 2pt minus 1pt}}
}{% if KOMA class
  \KOMAoptions{parskip=half}}
\makeatother
\usepackage{xcolor}
\IfFileExists{xurl.sty}{\usepackage{xurl}}{} % add URL line breaks if available
\IfFileExists{bookmark.sty}{\usepackage{bookmark}}{\usepackage{hyperref}}
\hypersetup{
  pdftitle={Tipología y ciclo de vida de los datos: Práctica 2},
  pdfauthor={Rubén Moya Vázquez rmoyav@uoc.edu},
  hidelinks,
  pdfcreator={LaTeX via pandoc}}
\urlstyle{same} % disable monospaced font for URLs
\usepackage[margin=1in]{geometry}
\usepackage{color}
\usepackage{fancyvrb}
\newcommand{\VerbBar}{|}
\newcommand{\VERB}{\Verb[commandchars=\\\{\}]}
\DefineVerbatimEnvironment{Highlighting}{Verbatim}{commandchars=\\\{\}}
% Add ',fontsize=\small' for more characters per line
\usepackage{framed}
\definecolor{shadecolor}{RGB}{48,48,48}
\newenvironment{Shaded}{\begin{snugshade}}{\end{snugshade}}
\newcommand{\AlertTok}[1]{\textcolor[rgb]{1.00,0.81,0.69}{#1}}
\newcommand{\AnnotationTok}[1]{\textcolor[rgb]{0.50,0.62,0.50}{\textbf{#1}}}
\newcommand{\AttributeTok}[1]{\textcolor[rgb]{0.80,0.80,0.80}{#1}}
\newcommand{\BaseNTok}[1]{\textcolor[rgb]{0.86,0.64,0.64}{#1}}
\newcommand{\BuiltInTok}[1]{\textcolor[rgb]{0.80,0.80,0.80}{#1}}
\newcommand{\CharTok}[1]{\textcolor[rgb]{0.86,0.64,0.64}{#1}}
\newcommand{\CommentTok}[1]{\textcolor[rgb]{0.50,0.62,0.50}{#1}}
\newcommand{\CommentVarTok}[1]{\textcolor[rgb]{0.50,0.62,0.50}{\textbf{#1}}}
\newcommand{\ConstantTok}[1]{\textcolor[rgb]{0.86,0.64,0.64}{\textbf{#1}}}
\newcommand{\ControlFlowTok}[1]{\textcolor[rgb]{0.94,0.87,0.69}{#1}}
\newcommand{\DataTypeTok}[1]{\textcolor[rgb]{0.87,0.87,0.75}{#1}}
\newcommand{\DecValTok}[1]{\textcolor[rgb]{0.86,0.86,0.80}{#1}}
\newcommand{\DocumentationTok}[1]{\textcolor[rgb]{0.50,0.62,0.50}{#1}}
\newcommand{\ErrorTok}[1]{\textcolor[rgb]{0.76,0.75,0.62}{#1}}
\newcommand{\ExtensionTok}[1]{\textcolor[rgb]{0.80,0.80,0.80}{#1}}
\newcommand{\FloatTok}[1]{\textcolor[rgb]{0.75,0.75,0.82}{#1}}
\newcommand{\FunctionTok}[1]{\textcolor[rgb]{0.94,0.94,0.56}{#1}}
\newcommand{\ImportTok}[1]{\textcolor[rgb]{0.80,0.80,0.80}{#1}}
\newcommand{\InformationTok}[1]{\textcolor[rgb]{0.50,0.62,0.50}{\textbf{#1}}}
\newcommand{\KeywordTok}[1]{\textcolor[rgb]{0.94,0.87,0.69}{#1}}
\newcommand{\NormalTok}[1]{\textcolor[rgb]{0.80,0.80,0.80}{#1}}
\newcommand{\OperatorTok}[1]{\textcolor[rgb]{0.94,0.94,0.82}{#1}}
\newcommand{\OtherTok}[1]{\textcolor[rgb]{0.94,0.94,0.56}{#1}}
\newcommand{\PreprocessorTok}[1]{\textcolor[rgb]{1.00,0.81,0.69}{\textbf{#1}}}
\newcommand{\RegionMarkerTok}[1]{\textcolor[rgb]{0.80,0.80,0.80}{#1}}
\newcommand{\SpecialCharTok}[1]{\textcolor[rgb]{0.86,0.64,0.64}{#1}}
\newcommand{\SpecialStringTok}[1]{\textcolor[rgb]{0.80,0.58,0.58}{#1}}
\newcommand{\StringTok}[1]{\textcolor[rgb]{0.80,0.58,0.58}{#1}}
\newcommand{\VariableTok}[1]{\textcolor[rgb]{0.80,0.80,0.80}{#1}}
\newcommand{\VerbatimStringTok}[1]{\textcolor[rgb]{0.80,0.58,0.58}{#1}}
\newcommand{\WarningTok}[1]{\textcolor[rgb]{0.50,0.62,0.50}{\textbf{#1}}}
\usepackage{graphicx}
\makeatletter
\def\maxwidth{\ifdim\Gin@nat@width>\linewidth\linewidth\else\Gin@nat@width\fi}
\def\maxheight{\ifdim\Gin@nat@height>\textheight\textheight\else\Gin@nat@height\fi}
\makeatother
% Scale images if necessary, so that they will not overflow the page
% margins by default, and it is still possible to overwrite the defaults
% using explicit options in \includegraphics[width, height, ...]{}
\setkeys{Gin}{width=\maxwidth,height=\maxheight,keepaspectratio}
% Set default figure placement to htbp
\makeatletter
\def\fps@figure{htbp}
\makeatother
\setlength{\emergencystretch}{3em} % prevent overfull lines
\providecommand{\tightlist}{%
  \setlength{\itemsep}{0pt}\setlength{\parskip}{0pt}}
\setcounter{secnumdepth}{-\maxdimen} % remove section numbering
\ifluatex
  \usepackage{selnolig}  % disable illegal ligatures
\fi

\title{Tipología y ciclo de vida de los datos: Práctica 2}
\author{Rubén Moya Vázquez
\href{mailto:rmoyav@uoc.edu}{\nolinkurl{rmoyav@uoc.edu}}}
\date{04/01/2022}

\begin{document}
\maketitle

{
\setcounter{tocdepth}{2}
\tableofcontents
}
\hypertarget{carga-de-datos}{%
\section{Carga de datos}\label{carga-de-datos}}

Comenzaremos por cargar los datos del csv en un dataframe, para ello
ejecutamos el siguiente código. Para simplificar el acceso a los datos,
renombraremos las columas de la siguiente manera: * Temperature..K.
-\textgreater{} temperature * Luminosity.L.Lo. -\textgreater{}
luminosity * Radius.R.Ro. -\textgreater{} radius *
Absolute.magnitude.Mv. -\textgreater{} magnitude * Star.type
-\textgreater{} type * Star.color -\textgreater{} color * Spectral.Class
-\textgreater{} spectral\_class

\begin{Shaded}
\begin{Highlighting}[]
\CommentTok{\# Cargamos los datos en nuestro dataset.}
\NormalTok{stars }\OtherTok{\textless{}{-}} \FunctionTok{read.csv}\NormalTok{(}\StringTok{\textquotesingle{}../data/6 \_class\_csv.csv\textquotesingle{}}\NormalTok{, }\AttributeTok{sep =} \StringTok{","}\NormalTok{,  }\AttributeTok{stringsAsFactors =} \ConstantTok{FALSE}\NormalTok{)}
\FunctionTok{names}\NormalTok{(stars) }\OtherTok{\textless{}{-}} \FunctionTok{c}\NormalTok{(}\StringTok{"temperature"}\NormalTok{, }\StringTok{"luminosity"}\NormalTok{, }\StringTok{"radius"}\NormalTok{, }\StringTok{"magnitude"}\NormalTok{, }\StringTok{"type"}\NormalTok{, }\StringTok{"color"}\NormalTok{, }\StringTok{"spectral\_class"}\NormalTok{)}
\CommentTok{\# Mostramos los primer y últimos 6 valores para observar que la carga}
\CommentTok{\# se ha realizado correctamente.}
\FunctionTok{head}\NormalTok{(stars)}
\end{Highlighting}
\end{Shaded}

\begin{verbatim}
##   temperature luminosity radius magnitude type color spectral_class
## 1        3068   0.002400 0.1700     16.12    0   Red              M
## 2        3042   0.000500 0.1542     16.60    0   Red              M
## 3        2600   0.000300 0.1020     18.70    0   Red              M
## 4        2800   0.000200 0.1600     16.65    0   Red              M
## 5        1939   0.000138 0.1030     20.06    0   Red              M
## 6        2840   0.000650 0.1100     16.98    0   Red              M
\end{verbatim}

\begin{Shaded}
\begin{Highlighting}[]
\FunctionTok{tail}\NormalTok{(stars)}
\end{Highlighting}
\end{Shaded}

\begin{verbatim}
##     temperature luminosity radius magnitude type      color spectral_class
## 235       21904     748490   1130     -7.67    5 Blue-white              B
## 236       38940     374830   1356     -9.93    5       Blue              O
## 237       30839     834042   1194    -10.63    5       Blue              O
## 238        8829     537493   1423    -10.73    5      White              A
## 239        9235     404940   1112    -11.23    5      White              A
## 240       37882     294903   1783     -7.80    5       Blue              O
\end{verbatim}

\hypertarget{limpieza-de-datos}{%
\section{Limpieza de datos}\label{limpieza-de-datos}}

A continuación, veremos la estructura de los datos y si es necesario
realizar una limpieza de valores nulos:

\begin{Shaded}
\begin{Highlighting}[]
\FunctionTok{str}\NormalTok{(stars)}
\end{Highlighting}
\end{Shaded}

\begin{verbatim}
## 'data.frame':    240 obs. of  7 variables:
##  $ temperature   : int  3068 3042 2600 2800 1939 2840 2637 2600 2650 2700 ...
##  $ luminosity    : num  0.0024 0.0005 0.0003 0.0002 0.000138 0.00065 0.00073 0.0004 0.00069 0.00018 ...
##  $ radius        : num  0.17 0.154 0.102 0.16 0.103 ...
##  $ magnitude     : num  16.1 16.6 18.7 16.6 20.1 ...
##  $ type          : int  0 0 0 0 0 0 0 0 0 0 ...
##  $ color         : chr  "Red" "Red" "Red" "Red" ...
##  $ spectral_class: chr  "M" "M" "M" "M" ...
\end{verbatim}

Vemos que tenemos dos variables enteras (la temperatura en grados kelvin
y el tipo de estrella), 3 variables de tipo numérico (radio, luminosidad
y magnitud) y 2 del tipo carácter (color y clase espectral).

\hypertarget{factorizaciuxf3n-de-columnas}{%
\subsection{Factorización de
columnas}\label{factorizaciuxf3n-de-columnas}}

Antes de limpiar los posibles valores nulos vamos a normalizar los
valores de las variables color y spectral\_class para convertirlas en
factores.

\begin{Shaded}
\begin{Highlighting}[]
\CommentTok{\# Buscamos los valores diferentes que puedan ser normalizados en la columna color}
\FunctionTok{unique}\NormalTok{(stars}\SpecialCharTok{$}\NormalTok{color)}
\end{Highlighting}
\end{Shaded}

\begin{verbatim}
##  [1] "Red"                "Blue White"         "White"             
##  [4] "Yellowish White"    "Blue white"         "Pale yellow orange"
##  [7] "Blue"               "Blue-white"         "Whitish"           
## [10] "yellow-white"       "Orange"             "White-Yellow"      
## [13] "white"              "Blue "              "yellowish"         
## [16] "Yellowish"          "Orange-Red"         "Blue white "       
## [19] "Blue-White"
\end{verbatim}

\begin{Shaded}
\begin{Highlighting}[]
\CommentTok{\# Vemos que hay varios valores que son asimilables al mismo pero con erratas tipográficas}
\NormalTok{stars}\SpecialCharTok{$}\NormalTok{color[stars}\SpecialCharTok{$}\NormalTok{color }\SpecialCharTok{==} \StringTok{"Blue White"}\NormalTok{] }\OtherTok{\textless{}{-}} \StringTok{"Blue{-}White"}
\NormalTok{stars}\SpecialCharTok{$}\NormalTok{color[stars}\SpecialCharTok{$}\NormalTok{color }\SpecialCharTok{==} \StringTok{"Blue white "}\NormalTok{] }\OtherTok{\textless{}{-}} \StringTok{"Blue{-}White"}
\NormalTok{stars}\SpecialCharTok{$}\NormalTok{color[stars}\SpecialCharTok{$}\NormalTok{color }\SpecialCharTok{==} \StringTok{"Blue white"}\NormalTok{] }\OtherTok{\textless{}{-}} \StringTok{"Blue{-}White"}
\NormalTok{stars}\SpecialCharTok{$}\NormalTok{color[stars}\SpecialCharTok{$}\NormalTok{color }\SpecialCharTok{==} \StringTok{"Blue{-}white"}\NormalTok{] }\OtherTok{\textless{}{-}} \StringTok{"Blue{-}White"}
\NormalTok{stars}\SpecialCharTok{$}\NormalTok{color[stars}\SpecialCharTok{$}\NormalTok{color }\SpecialCharTok{==} \StringTok{"Blue "}\NormalTok{] }\OtherTok{\textless{}{-}} \StringTok{"Blue"}
\NormalTok{stars}\SpecialCharTok{$}\NormalTok{color[stars}\SpecialCharTok{$}\NormalTok{color }\SpecialCharTok{==} \StringTok{"white"}\NormalTok{] }\OtherTok{\textless{}{-}} \StringTok{"White"}
\NormalTok{stars}\SpecialCharTok{$}\NormalTok{color[stars}\SpecialCharTok{$}\NormalTok{color }\SpecialCharTok{==} \StringTok{"yellowish"}\NormalTok{] }\OtherTok{\textless{}{-}} \StringTok{"Yellowish"}
\NormalTok{stars}\SpecialCharTok{$}\NormalTok{color[stars}\SpecialCharTok{$}\NormalTok{color }\SpecialCharTok{==} \StringTok{"yellow{-}white"}\NormalTok{] }\OtherTok{\textless{}{-}} \StringTok{"White{-}Yellow"}
\NormalTok{stars}\SpecialCharTok{$}\NormalTok{color[stars}\SpecialCharTok{$}\NormalTok{color }\SpecialCharTok{==} \StringTok{"Yellowish White"}\NormalTok{] }\OtherTok{\textless{}{-}} \StringTok{"White{-}Yellow"}
\NormalTok{stars}\SpecialCharTok{$}\NormalTok{color[stars}\SpecialCharTok{$}\NormalTok{color }\SpecialCharTok{==} \StringTok{"Pale yellow orange"}\NormalTok{] }\OtherTok{\textless{}{-}} \StringTok{"Yellow{-}Orange"}

\CommentTok{\# Convertimos la columna en factor}
\NormalTok{stars}\SpecialCharTok{$}\NormalTok{color }\OtherTok{\textless{}{-}} \FunctionTok{as.factor}\NormalTok{(stars}\SpecialCharTok{$}\NormalTok{color)}

\CommentTok{\# Convertimos la columna type en factor}
\NormalTok{stars}\SpecialCharTok{$}\NormalTok{type }\OtherTok{\textless{}{-}} \FunctionTok{factor}\NormalTok{(stars}\SpecialCharTok{$}\NormalTok{type, }\AttributeTok{levels =} \FunctionTok{c}\NormalTok{(}\DecValTok{0}\NormalTok{, }\DecValTok{1}\NormalTok{, }\DecValTok{2}\NormalTok{, }\DecValTok{3}\NormalTok{, }\DecValTok{4}\NormalTok{, }\DecValTok{5}\NormalTok{), }\AttributeTok{labels =} \FunctionTok{c}\NormalTok{(}\StringTok{"Brown Dwarf"}\NormalTok{, }\StringTok{"Red Dwarf"}\NormalTok{, }\StringTok{"White Dwarf"}\NormalTok{, }\StringTok{"Main Sequence"}\NormalTok{, }\StringTok{"Supergiant"}\NormalTok{, }\StringTok{"Hypergiant"}\NormalTok{))}
\end{Highlighting}
\end{Shaded}

\begin{Shaded}
\begin{Highlighting}[]
\CommentTok{\# Hacemos lo mismo con la columna spectral\_class}
\FunctionTok{unique}\NormalTok{(stars}\SpecialCharTok{$}\NormalTok{spectral\_class)}
\end{Highlighting}
\end{Shaded}

\begin{verbatim}
## [1] "M" "B" "A" "F" "O" "K" "G"
\end{verbatim}

\begin{Shaded}
\begin{Highlighting}[]
\CommentTok{\# En este caso observamos que los valores son los correctos, asi que factorizamos}
\NormalTok{stars}\SpecialCharTok{$}\NormalTok{spectral\_class }\OtherTok{\textless{}{-}} \FunctionTok{as.factor}\NormalTok{(stars}\SpecialCharTok{$}\NormalTok{spectral\_class)}
\end{Highlighting}
\end{Shaded}

\hypertarget{gestiuxf3n-de-valores-nulos}{%
\subsection{Gestión de valores
nulos}\label{gestiuxf3n-de-valores-nulos}}

Ahora pasaremos a comprobar si hay valores nulos en alguna de las
columnas de nuestro conjunto de datos. Haremos la comprobación en las
columnas numéricas y en las enteras, ya que tanto en type como en las
columnas factores sabemos que no hay.

\begin{Shaded}
\begin{Highlighting}[]
\FunctionTok{any}\NormalTok{(}\FunctionTok{is.na}\NormalTok{(stars}\SpecialCharTok{$}\NormalTok{temperature))}
\end{Highlighting}
\end{Shaded}

\begin{verbatim}
## [1] FALSE
\end{verbatim}

\begin{Shaded}
\begin{Highlighting}[]
\FunctionTok{any}\NormalTok{(}\FunctionTok{is.na}\NormalTok{(stars}\SpecialCharTok{$}\NormalTok{luminosity))}
\end{Highlighting}
\end{Shaded}

\begin{verbatim}
## [1] FALSE
\end{verbatim}

\begin{Shaded}
\begin{Highlighting}[]
\FunctionTok{any}\NormalTok{(}\FunctionTok{is.na}\NormalTok{(stars}\SpecialCharTok{$}\NormalTok{radius))}
\end{Highlighting}
\end{Shaded}

\begin{verbatim}
## [1] FALSE
\end{verbatim}

\begin{Shaded}
\begin{Highlighting}[]
\FunctionTok{any}\NormalTok{(}\FunctionTok{is.na}\NormalTok{(stars}\SpecialCharTok{$}\NormalTok{magnitude))}
\end{Highlighting}
\end{Shaded}

\begin{verbatim}
## [1] FALSE
\end{verbatim}

Estos cuatro valores ``FALSE'' nos indican que no hay ningún valor nulo
en nuestro conjunto de datos. De todas formas, eso es algo que ya nos
indicaban en la información del dataset de Kaggle.

\hypertarget{gestiuxf3n-de-valores-extremos.}{%
\subsection{Gestión de valores
extremos.}\label{gestiuxf3n-de-valores-extremos.}}

En primer lugar, utilizaremos diagramas de cajas para buscar los
posibles valores extremos en las columnas de nuestro dataset.

\begin{Shaded}
\begin{Highlighting}[]
\FunctionTok{boxplot}\NormalTok{(stars}\SpecialCharTok{$}\NormalTok{temperature, }\AttributeTok{main =} \StringTok{"Temperature"}\NormalTok{, }\AttributeTok{col =} \StringTok{"orange"}\NormalTok{, }\AttributeTok{border =} \StringTok{"brown"}\NormalTok{, }\AttributeTok{horizontal =} \ConstantTok{TRUE}\NormalTok{, }\AttributeTok{notch =} \ConstantTok{TRUE}\NormalTok{)}
\end{Highlighting}
\end{Shaded}

\includegraphics{star_cleaner_files/figure-latex/unnamed-chunk-6-1.pdf}

\begin{Shaded}
\begin{Highlighting}[]
\FunctionTok{boxplot}\NormalTok{(stars}\SpecialCharTok{$}\NormalTok{luminosity,  }\AttributeTok{main =} \StringTok{"Luminosity"}\NormalTok{, }\AttributeTok{col =} \StringTok{"orange"}\NormalTok{, }\AttributeTok{border =} \StringTok{"brown"}\NormalTok{, }\AttributeTok{horizontal =} \ConstantTok{TRUE}\NormalTok{, }\AttributeTok{notch =} \ConstantTok{TRUE}\NormalTok{)}
\end{Highlighting}
\end{Shaded}

\includegraphics{star_cleaner_files/figure-latex/unnamed-chunk-6-2.pdf}

\begin{Shaded}
\begin{Highlighting}[]
\FunctionTok{boxplot}\NormalTok{(stars}\SpecialCharTok{$}\NormalTok{radius,  }\AttributeTok{main =} \StringTok{"Radius"}\NormalTok{, }\AttributeTok{col =} \StringTok{"orange"}\NormalTok{, }\AttributeTok{border =} \StringTok{"brown"}\NormalTok{, }\AttributeTok{horizontal =} \ConstantTok{TRUE}\NormalTok{, }\AttributeTok{notch =} \ConstantTok{TRUE}\NormalTok{)}
\end{Highlighting}
\end{Shaded}

\includegraphics{star_cleaner_files/figure-latex/unnamed-chunk-6-3.pdf}

\begin{Shaded}
\begin{Highlighting}[]
\FunctionTok{boxplot}\NormalTok{(stars}\SpecialCharTok{$}\NormalTok{magnitude,  }\AttributeTok{main =} \StringTok{"Magnitude"}\NormalTok{, }\AttributeTok{col =} \StringTok{"orange"}\NormalTok{, }\AttributeTok{border =} \StringTok{"brown"}\NormalTok{, }\AttributeTok{horizontal =} \ConstantTok{TRUE}\NormalTok{, }\AttributeTok{notch =} \ConstantTok{TRUE}\NormalTok{)}
\end{Highlighting}
\end{Shaded}

\includegraphics{star_cleaner_files/figure-latex/unnamed-chunk-6-4.pdf}

Las gráficas mostradas nos hacen pensar que en los tres primeros casos
podríamos encontrarnos ante distribuciones log-normales, mientras que en
la ultima, no observamos ningún outlier. Para asegurarnos de que en los
casos anteriores nos hayamos ante distribuciones log-normales,
estudiaremos sus histogramas.

\begin{Shaded}
\begin{Highlighting}[]
\FunctionTok{hist}\NormalTok{(}\FunctionTok{log}\NormalTok{(stars}\SpecialCharTok{$}\NormalTok{temperature), }\AttributeTok{breaks =} \DecValTok{16}\NormalTok{, }\AttributeTok{prob =} \ConstantTok{TRUE}\NormalTok{)}
\FunctionTok{lines}\NormalTok{(}\FunctionTok{density}\NormalTok{(}\FunctionTok{log}\NormalTok{(stars}\SpecialCharTok{$}\NormalTok{temperature)), }\AttributeTok{lwd =} \DecValTok{4}\NormalTok{, }\AttributeTok{col =} \StringTok{"chocolate3"}\NormalTok{)}
\end{Highlighting}
\end{Shaded}

\includegraphics{star_cleaner_files/figure-latex/unnamed-chunk-7-1.pdf}

\begin{Shaded}
\begin{Highlighting}[]
\FunctionTok{hist}\NormalTok{(}\FunctionTok{log}\NormalTok{(stars}\SpecialCharTok{$}\NormalTok{luminosity), }\AttributeTok{breaks =} \DecValTok{16}\NormalTok{, }\AttributeTok{prob =} \ConstantTok{TRUE}\NormalTok{)}
\FunctionTok{lines}\NormalTok{(}\FunctionTok{density}\NormalTok{(}\FunctionTok{log}\NormalTok{(stars}\SpecialCharTok{$}\NormalTok{luminosity)), }\AttributeTok{lwd =} \DecValTok{4}\NormalTok{, }\AttributeTok{col =} \StringTok{"chocolate3"}\NormalTok{)}
\end{Highlighting}
\end{Shaded}

\includegraphics{star_cleaner_files/figure-latex/unnamed-chunk-7-2.pdf}

\begin{Shaded}
\begin{Highlighting}[]
\FunctionTok{hist}\NormalTok{(}\FunctionTok{log}\NormalTok{(stars}\SpecialCharTok{$}\NormalTok{radius), }\AttributeTok{breaks =} \DecValTok{16}\NormalTok{, }\AttributeTok{prob =} \ConstantTok{TRUE}\NormalTok{)}
\FunctionTok{lines}\NormalTok{(}\FunctionTok{density}\NormalTok{(}\FunctionTok{log}\NormalTok{(stars}\SpecialCharTok{$}\NormalTok{radius)), }\AttributeTok{lwd =} \DecValTok{4}\NormalTok{, }\AttributeTok{col =} \StringTok{"chocolate3"}\NormalTok{)}
\end{Highlighting}
\end{Shaded}

\includegraphics{star_cleaner_files/figure-latex/unnamed-chunk-7-3.pdf}

Como podemos ver, ninguna de las distribuciones sigue la estructura de
campana de Gauss presente en una distribución normal, así que podemos
descartar que sean distribuciones log-normales sin necesidad de realizar
más pruebas.

Finalmente, en cuanto a los valores extremos, hemos decidido
contemplarlos en nuestro análisis sin modificarlos. Esta decisión se ha
tomado teniendo en cuenta que dichos valores responden a mediciones
reales que, dado el pequeño tamaño de la muestra, pueden ser demasiado
variadas como para mostrar una distribución normal. Además, consideramos
que, en este punto, desconocemos la relevancia de dichos valores
extremos como para descartarlos o modificarlos de antemano.

\hypertarget{anuxe1lisis-de-los-datos}{%
\section{Análisis de los Datos}\label{anuxe1lisis-de-los-datos}}

\hypertarget{selecciuxf3n-de-los-grupos-de-datos-que-se-quieren-analizar}{%
\subsection{Selección de los grupos de datos que se quieren
analizar}\label{selecciuxf3n-de-los-grupos-de-datos-que-se-quieren-analizar}}

A continuación agruparemos los datos en función de diversos criterios
para formar subconjuntos que puedan resultar interesantes para su
análisis. Realizaremos varios conjuntos y no necesariamente todos ellos
serán utilizados durante el análisis final.

\begin{Shaded}
\begin{Highlighting}[]
\CommentTok{\# Por tipo de estrella}
\NormalTok{stars.brown\_dwarfs }\OtherTok{\textless{}{-}}\NormalTok{ stars[stars}\SpecialCharTok{$}\NormalTok{type }\SpecialCharTok{==} \DecValTok{0}\NormalTok{, ]}
\NormalTok{stars.red\_dwarfs }\OtherTok{\textless{}{-}}\NormalTok{ stars[stars}\SpecialCharTok{$}\NormalTok{type }\SpecialCharTok{==} \DecValTok{1}\NormalTok{, ]}
\NormalTok{stars.white\_dwarfs }\OtherTok{\textless{}{-}}\NormalTok{ stars[stars}\SpecialCharTok{$}\NormalTok{type }\SpecialCharTok{==} \DecValTok{2}\NormalTok{, ]}
\NormalTok{stars.main\_sequence }\OtherTok{\textless{}{-}}\NormalTok{ stars[stars}\SpecialCharTok{$}\NormalTok{type }\SpecialCharTok{==} \DecValTok{3}\NormalTok{, ]}
\NormalTok{stars.supergiant }\OtherTok{\textless{}{-}}\NormalTok{ stars[stars}\SpecialCharTok{$}\NormalTok{type }\SpecialCharTok{==} \DecValTok{4}\NormalTok{, ]}
\NormalTok{stars.hypergiant }\OtherTok{\textless{}{-}}\NormalTok{ stars[stars}\SpecialCharTok{$}\NormalTok{type }\SpecialCharTok{==} \DecValTok{5}\NormalTok{, ]}

\CommentTok{\# Por clase espectral}
\NormalTok{stars.O }\OtherTok{\textless{}{-}}\NormalTok{ stars[stars}\SpecialCharTok{$}\NormalTok{spectral\_class }\SpecialCharTok{==} \StringTok{"O"}\NormalTok{, ]}
\NormalTok{stars.B }\OtherTok{\textless{}{-}}\NormalTok{ stars[stars}\SpecialCharTok{$}\NormalTok{spectral\_class }\SpecialCharTok{==} \StringTok{"B"}\NormalTok{, ]}
\NormalTok{stars.A }\OtherTok{\textless{}{-}}\NormalTok{ stars[stars}\SpecialCharTok{$}\NormalTok{spectral\_class }\SpecialCharTok{==} \StringTok{"A"}\NormalTok{, ]}
\NormalTok{stars.F }\OtherTok{\textless{}{-}}\NormalTok{ stars[stars}\SpecialCharTok{$}\NormalTok{spectral\_class }\SpecialCharTok{==} \StringTok{"F"}\NormalTok{, ]}
\NormalTok{stars.G }\OtherTok{\textless{}{-}}\NormalTok{ stars[stars}\SpecialCharTok{$}\NormalTok{spectral\_class }\SpecialCharTok{==} \StringTok{"G"}\NormalTok{, ]}
\NormalTok{stars.K }\OtherTok{\textless{}{-}}\NormalTok{ stars[stars}\SpecialCharTok{$}\NormalTok{spectral\_class }\SpecialCharTok{==} \StringTok{"K"}\NormalTok{, ]}
\NormalTok{stars.M }\OtherTok{\textless{}{-}}\NormalTok{ stars[stars}\SpecialCharTok{$}\NormalTok{spectral\_class }\SpecialCharTok{==} \StringTok{"M"}\NormalTok{, ]}

\CommentTok{\# Por color}
\NormalTok{stars.red }\OtherTok{\textless{}{-}}\NormalTok{ stars[stars}\SpecialCharTok{$}\NormalTok{color }\SpecialCharTok{==} \StringTok{"Red"}\NormalTok{, ]}
\NormalTok{stars.blue }\OtherTok{\textless{}{-}}\NormalTok{ stars[stars}\SpecialCharTok{$}\NormalTok{color }\SpecialCharTok{==} \StringTok{"Blue"}\NormalTok{, ]}
\NormalTok{stars.white }\OtherTok{\textless{}{-}}\NormalTok{ stars[stars}\SpecialCharTok{$}\NormalTok{color }\SpecialCharTok{==} \StringTok{"White"}\NormalTok{, ]}
\end{Highlighting}
\end{Shaded}

\hypertarget{comprobaciuxf3n-de-la-normalidad-y-homogeneidad-de-la-varianza}{%
\subsection{Comprobación de la normalidad y homogeneidad de la
varianza}\label{comprobaciuxf3n-de-la-normalidad-y-homogeneidad-de-la-varianza}}

\hypertarget{estudio-de-la-normalidad}{%
\subsubsection{Estudio de la
normalidad}\label{estudio-de-la-normalidad}}

Vamos a estudiar brevemente si alguna de nuestras variables numéricas
sigue una distribución normal. Para ello, nos serviremos de la prueba de
normalidad de Anderson-Darling.

\begin{Shaded}
\begin{Highlighting}[]
\FunctionTok{library}\NormalTok{(nortest)}

\NormalTok{alpha }\OtherTok{=} \FloatTok{0.05}
\NormalTok{col.names }\OtherTok{=} \FunctionTok{colnames}\NormalTok{(stars)}
\FunctionTok{print}\NormalTok{(}\StringTok{"Las variables numéricas que no siguen una distribución normal son:"}\NormalTok{)}
\end{Highlighting}
\end{Shaded}

\begin{verbatim}
## [1] "Las variables numéricas que no siguen una distribución normal son:"
\end{verbatim}

\begin{Shaded}
\begin{Highlighting}[]
\ControlFlowTok{for}\NormalTok{ (i }\ControlFlowTok{in} \DecValTok{1}\SpecialCharTok{:}\FunctionTok{ncol}\NormalTok{(stars)) \{}
  \ControlFlowTok{if}\NormalTok{ (}\FunctionTok{is.numeric}\NormalTok{(stars[,i])) \{}
\NormalTok{    p\_val }\OtherTok{=} \FunctionTok{ad.test}\NormalTok{(stars[,i])}\SpecialCharTok{$}\NormalTok{p.value}
    \ControlFlowTok{if}\NormalTok{ (p\_val }\SpecialCharTok{\textless{}}\NormalTok{ alpha) \{}
      \FunctionTok{print}\NormalTok{(col.names[i])}
\NormalTok{    \}}
\NormalTok{  \}}
\NormalTok{\}}
\end{Highlighting}
\end{Shaded}

\begin{verbatim}
## [1] "temperature"
## [1] "luminosity"
## [1] "radius"
## [1] "magnitude"
\end{verbatim}

Es decir, ninguna de nuestras variables numéricas sigue una distribución
normal, pero\ldots{} ¿Podrían seguirla?

Vamos a analizar si alguna de nuestras variables podría ser candidata a
la normalización mediante el estudio de las gráficas quantile-quantile y
su histograma.

\begin{Shaded}
\begin{Highlighting}[]
\FunctionTok{par}\NormalTok{(}\AttributeTok{mfrow=}\FunctionTok{c}\NormalTok{(}\DecValTok{3}\NormalTok{,}\DecValTok{2}\NormalTok{))}
\ControlFlowTok{for}\NormalTok{(i }\ControlFlowTok{in} \DecValTok{1}\SpecialCharTok{:}\FunctionTok{ncol}\NormalTok{(stars)) \{}
  \ControlFlowTok{if}\NormalTok{ (}\FunctionTok{is.numeric}\NormalTok{(stars[,i]))\{}
    \FunctionTok{qqnorm}\NormalTok{(stars[,i],}\AttributeTok{main =} \FunctionTok{paste}\NormalTok{(}\StringTok{"Quantile{-}Quantile de"}\NormalTok{,}\FunctionTok{colnames}\NormalTok{(stars)[i]))}
    \FunctionTok{qqline}\NormalTok{(stars[,i],}\AttributeTok{col=}\StringTok{"red"}\NormalTok{)}
    \FunctionTok{hist}\NormalTok{(stars[,i], }
      \AttributeTok{main=}\FunctionTok{paste}\NormalTok{(}\StringTok{"Histograma de"}\NormalTok{, }\FunctionTok{colnames}\NormalTok{(stars)[i]), }
      \AttributeTok{xlab=}\FunctionTok{colnames}\NormalTok{(stars)[i], }\AttributeTok{freq =} \ConstantTok{FALSE}\NormalTok{)}
\NormalTok{  \}}
\NormalTok{\}}
\end{Highlighting}
\end{Shaded}

\includegraphics{star_cleaner_files/figure-latex/unnamed-chunk-10-1.pdf}
\includegraphics{star_cleaner_files/figure-latex/unnamed-chunk-10-2.pdf}

Estos resultados nos muestran que varias de nuestras variables numéricas
pueden ser candidatas a la normalización. De todas formas, descartaremos
este proceso debido a que queremos que se estudien los datos reales de
las mediciones de nuestro conjunto de estrellas.

\hypertarget{estudio-de-la-homogeneidad-de-la-varianza}{%
\subsubsection{Estudio de la homogeneidad de la
varianza}\label{estudio-de-la-homogeneidad-de-la-varianza}}

Para estudiar la homogeneidad de la varianza utilizaremos el test de
Levene que nos permite realizar comparaciones de varianzas con 2 o más
conjuntos de elementos.

\begin{Shaded}
\begin{Highlighting}[]
\FunctionTok{library}\NormalTok{(car)}
\FunctionTok{leveneTest}\NormalTok{(}\AttributeTok{y =}\NormalTok{ stars}\SpecialCharTok{$}\NormalTok{temperature, }\AttributeTok{group =}\NormalTok{ stars}\SpecialCharTok{$}\NormalTok{type, }\AttributeTok{center =} \StringTok{"median"}\NormalTok{)}
\end{Highlighting}
\end{Shaded}

\begin{verbatim}
## Levene's Test for Homogeneity of Variance (center = "median")
##        Df F value    Pr(>F)    
## group   5  16.063 1.316e-13 ***
##       234                      
## ---
## Signif. codes:  0 '***' 0.001 '**' 0.01 '*' 0.05 '.' 0.1 ' ' 1
\end{verbatim}

\begin{Shaded}
\begin{Highlighting}[]
\FunctionTok{leveneTest}\NormalTok{(}\AttributeTok{y =}\NormalTok{ stars}\SpecialCharTok{$}\NormalTok{luminosity, }\AttributeTok{group =}\NormalTok{ stars}\SpecialCharTok{$}\NormalTok{type, }\AttributeTok{center =} \StringTok{"median"}\NormalTok{)}
\end{Highlighting}
\end{Shaded}

\begin{verbatim}
## Levene's Test for Homogeneity of Variance (center = "median")
##        Df F value    Pr(>F)    
## group   5  20.122 < 2.2e-16 ***
##       234                      
## ---
## Signif. codes:  0 '***' 0.001 '**' 0.01 '*' 0.05 '.' 0.1 ' ' 1
\end{verbatim}

\begin{Shaded}
\begin{Highlighting}[]
\FunctionTok{leveneTest}\NormalTok{(}\AttributeTok{y =}\NormalTok{ stars}\SpecialCharTok{$}\NormalTok{radius, }\AttributeTok{group =}\NormalTok{ stars}\SpecialCharTok{$}\NormalTok{type, }\AttributeTok{center =} \StringTok{"median"}\NormalTok{)}
\end{Highlighting}
\end{Shaded}

\begin{verbatim}
## Levene's Test for Homogeneity of Variance (center = "median")
##        Df F value    Pr(>F)    
## group   5  59.322 < 2.2e-16 ***
##       234                      
## ---
## Signif. codes:  0 '***' 0.001 '**' 0.01 '*' 0.05 '.' 0.1 ' ' 1
\end{verbatim}

\begin{Shaded}
\begin{Highlighting}[]
\FunctionTok{leveneTest}\NormalTok{(}\AttributeTok{y =}\NormalTok{ stars}\SpecialCharTok{$}\NormalTok{magnitude, }\AttributeTok{group =}\NormalTok{ stars}\SpecialCharTok{$}\NormalTok{type, }\AttributeTok{center =} \StringTok{"median"}\NormalTok{)}
\end{Highlighting}
\end{Shaded}

\begin{verbatim}
## Levene's Test for Homogeneity of Variance (center = "median")
##        Df F value    Pr(>F)    
## group   5  30.877 < 2.2e-16 ***
##       234                      
## ---
## Signif. codes:  0 '***' 0.001 '**' 0.01 '*' 0.05 '.' 0.1 ' ' 1
\end{verbatim}

Vemos que los resultados de estos test nos hacen percatarnos de que las
varianzas de nuestros grupos de datos no son homogeneas en todos los
casos.

\hypertarget{pruebas-estaduxedsticas}{%
\section{Pruebas estadísticas}\label{pruebas-estaduxedsticas}}

\hypertarget{las-estrellas-de-la-secuencia-principal-son-muxe1s-luminosas-que-las-enanas-blancas}{%
\subsection{¿Las estrellas de la secuencia principal son más luminosas
que las enanas
blancas?}\label{las-estrellas-de-la-secuencia-principal-son-muxe1s-luminosas-que-las-enanas-blancas}}

A continuación vamos a realizar un contraste de hipotesis para confirmar
o descartar el hecho de que las estrellas de la secuencia principal sean
más luminosas que las enanas blancas. Para realizar este contraste
utilizaremos los grupos que hemos creado anteriormente.

En nuestro caso, planteamos el contraste de hipótesis de dos muestras
sobre la diferencia de medias, unilatera, con un nivel de confianza del
95\%

\textbf{H0: μ1 − μ2 = 0}

\textbf{H1: μ1 − μ2 \textless{} 0}

\begin{Shaded}
\begin{Highlighting}[]
\CommentTok{\# Comentado para poder hacer el knit}
\CommentTok{\#t.test(stars.main\_sequence$luminosity, stars.white\_dwarfs$luminosity, alternative = "less")}
\end{Highlighting}
\end{Shaded}

Como podemos ver, no tenemos suficientes muestras para realizar dicho
contraste de hipotesis.

\hypertarget{random-forest-con-k-fold-cross-validation}{%
\subsection{Random forest con k-fold cross
validation}\label{random-forest-con-k-fold-cross-validation}}

A continuación vamos a aplicar un modelo de predicción conocido por su
efectividad como es el Random forest con la metodología k-fold de
validación cruzada, de manera que nos permita elaborar un modelo de
predicción eficaz para la clasificación de estrellas en base a nuestro
conjunto de datos. Gracias al uso de la validación cruzada nos
evitaremos el problema de tener un algoritmo sobreentrenado, lo cual es
un problema comun en este tipo de modelos.

\begin{Shaded}
\begin{Highlighting}[]
\FunctionTok{library}\NormalTok{(randomForest)}
\FunctionTok{library}\NormalTok{(caTools)}
\FunctionTok{library}\NormalTok{(caret)}

\FunctionTok{set.seed}\NormalTok{(}\DecValTok{12}\NormalTok{)}

\CommentTok{\# Creamos los grupos de entrenamiento y test}
\NormalTok{split }\OtherTok{\textless{}{-}} \FunctionTok{sample.split}\NormalTok{(stars}\SpecialCharTok{$}\NormalTok{type, }\AttributeTok{SplitRatio =} \FloatTok{0.80}\NormalTok{)}
\NormalTok{training\_set }\OtherTok{\textless{}{-}} \FunctionTok{subset}\NormalTok{(stars, split }\SpecialCharTok{==} \ConstantTok{TRUE}\NormalTok{)}
\NormalTok{test\_set }\OtherTok{\textless{}{-}} \FunctionTok{subset}\NormalTok{(stars, split }\SpecialCharTok{==} \ConstantTok{FALSE}\NormalTok{)}
\NormalTok{folds }\OtherTok{\textless{}{-}} \FunctionTok{createFolds}\NormalTok{(training\_set}\SpecialCharTok{$}\NormalTok{type, }\AttributeTok{k =} \DecValTok{10}\NormalTok{)}

\CommentTok{\# Creamos nuestro random forest con cross validation}
\NormalTok{cvRandomForest }\OtherTok{\textless{}{-}} \FunctionTok{lapply}\NormalTok{(folds, }\ControlFlowTok{function}\NormalTok{(x)\{}
\NormalTok{  training\_fold }\OtherTok{\textless{}{-}}\NormalTok{ training\_set[}\SpecialCharTok{{-}}\NormalTok{x, ]}
\NormalTok{  test\_fold }\OtherTok{\textless{}{-}}\NormalTok{ training\_set[x, ]}
\NormalTok{  clasifier }\OtherTok{\textless{}{-}} \FunctionTok{randomForest}\NormalTok{(type }\SpecialCharTok{\textasciitilde{}}\NormalTok{ ., }\AttributeTok{data =}\NormalTok{ training\_fold, }\AttributeTok{ntree =} \DecValTok{25}\NormalTok{)}
\NormalTok{  y\_pred }\OtherTok{\textless{}{-}} \FunctionTok{predict}\NormalTok{(clasifier, }\AttributeTok{newdata =}\NormalTok{ test\_fold)}
\NormalTok{  cm }\OtherTok{\textless{}{-}} \FunctionTok{table}\NormalTok{(test\_fold}\SpecialCharTok{$}\NormalTok{type, y\_pred)}
\NormalTok{  precision }\OtherTok{\textless{}{-}}\NormalTok{ (cm[}\DecValTok{1}\NormalTok{,}\DecValTok{1}\NormalTok{] }\SpecialCharTok{+}\NormalTok{ cm[}\DecValTok{2}\NormalTok{,}\DecValTok{2}\NormalTok{]) }\SpecialCharTok{/}\NormalTok{ (cm[}\DecValTok{1}\NormalTok{,}\DecValTok{1}\NormalTok{] }\SpecialCharTok{+}\NormalTok{ cm[}\DecValTok{2}\NormalTok{,}\DecValTok{2}\NormalTok{] }\SpecialCharTok{+}\NormalTok{cm[}\DecValTok{1}\NormalTok{,}\DecValTok{2}\NormalTok{] }\SpecialCharTok{+}\NormalTok{ cm[}\DecValTok{2}\NormalTok{,}\DecValTok{1}\NormalTok{])}
  \FunctionTok{return}\NormalTok{(precision)}
\NormalTok{\})}

\CommentTok{\# Precisión obtenida}
\NormalTok{precisionRandomForest }\OtherTok{\textless{}{-}} \FunctionTok{mean}\NormalTok{(}\FunctionTok{as.numeric}\NormalTok{(cvRandomForest))}
\FunctionTok{print}\NormalTok{(}\StringTok{"Precisión de nuestro modelo:"}\NormalTok{)}
\end{Highlighting}
\end{Shaded}

\begin{verbatim}
## [1] "Precisión de nuestro modelo:"
\end{verbatim}

\begin{Shaded}
\begin{Highlighting}[]
\NormalTok{precisionRandomForest}
\end{Highlighting}
\end{Shaded}

\begin{verbatim}
## [1] 0.9857143
\end{verbatim}

Como podemos ver, el modelo final tiene una precisión muy alta: 98.57\%.

Ahora visualizaremos de manera un poco más grafica los resultados de
predecir clasificaciones con nuestro modelo gracias a su matriz de
confusión:

\begin{Shaded}
\begin{Highlighting}[]
\NormalTok{clasificadorRF }\OtherTok{\textless{}{-}} \FunctionTok{randomForest}\NormalTok{(type }\SpecialCharTok{\textasciitilde{}}\NormalTok{ ., }\AttributeTok{data =}\NormalTok{ training\_set, }\AttributeTok{ntree =} \DecValTok{25}\NormalTok{)}
\NormalTok{y\_pred }\OtherTok{\textless{}{-}} \FunctionTok{predict}\NormalTok{(clasificadorRF, }\AttributeTok{newdata =}\NormalTok{ test\_set)}
\NormalTok{cm }\OtherTok{\textless{}{-}} \FunctionTok{table}\NormalTok{(test\_set}\SpecialCharTok{$}\NormalTok{type, y\_pred)}
\NormalTok{cm}
\end{Highlighting}
\end{Shaded}

\begin{verbatim}
##                y_pred
##                 Brown Dwarf Red Dwarf White Dwarf Main Sequence Supergiant
##   Brown Dwarf             8         0           0             0          0
##   Red Dwarf               0         8           0             0          0
##   White Dwarf             0         0           8             0          0
##   Main Sequence           0         0           0             8          0
##   Supergiant              0         0           0             0          8
##   Hypergiant              0         0           0             0          0
##                y_pred
##                 Hypergiant
##   Brown Dwarf            0
##   Red Dwarf              0
##   White Dwarf            0
##   Main Sequence          0
##   Supergiant             0
##   Hypergiant             8
\end{verbatim}

\hypertarget{estudio-del-impacto-del-radio-sobre-la-categorizaciuxf3n-de-una-estrella.}{%
\subsection{Estudio del impacto del radio sobre la categorización de una
estrella.}\label{estudio-del-impacto-del-radio-sobre-la-categorizaciuxf3n-de-una-estrella.}}

Para acabar vamos a realizar un pequeño estudio de la influencia que
tiene el radio (la variable de la que más dudamos) sobre la
categorización final de una estrella. Para ello utilizaremos un análisis
de correlación entre la variable y el tipo. Con el objetivo de medir
dicha correlación, aplicaremos regresión logística.

\hypertarget{regresiuxf3n-logistica.}{%
\subsubsection{Regresión logistica.}\label{regresiuxf3n-logistica.}}

Comenzamos por el estudio de las variables numéricas.

\begin{Shaded}
\begin{Highlighting}[]
\FunctionTok{library}\NormalTok{(caTools)}

\CommentTok{\# Dividimos el conjunto de datos}
\NormalTok{split\_reg }\OtherTok{\textless{}{-}} \FunctionTok{sample.split}\NormalTok{(stars, }\AttributeTok{SplitRatio =} \FloatTok{0.8}\NormalTok{)}
   
\NormalTok{train\_reg }\OtherTok{\textless{}{-}} \FunctionTok{subset}\NormalTok{(stars, split }\SpecialCharTok{==} \StringTok{"TRUE"}\NormalTok{)}
\NormalTok{test\_reg }\OtherTok{\textless{}{-}} \FunctionTok{subset}\NormalTok{(stars, split }\SpecialCharTok{==} \StringTok{"FALSE"}\NormalTok{)}
   
\CommentTok{\# Training model}
\NormalTok{logistic\_model }\OtherTok{\textless{}{-}} \FunctionTok{glm}\NormalTok{(type }\SpecialCharTok{\textasciitilde{}}\NormalTok{ radius, }
                      \AttributeTok{data =}\NormalTok{ train\_reg, }
                      \AttributeTok{family =} \StringTok{"binomial"}\NormalTok{)}
\NormalTok{logistic\_model}
\end{Highlighting}
\end{Shaded}

\begin{verbatim}
## 
## Call:  glm(formula = type ~ radius, family = "binomial", data = train_reg)
## 
## Coefficients:
## (Intercept)       radius  
##      0.1926       3.7328  
## 
## Degrees of Freedom: 191 Total (i.e. Null);  190 Residual
## Null Deviance:       173 
## Residual Deviance: 118   AIC: 122
\end{verbatim}

\begin{Shaded}
\begin{Highlighting}[]
\CommentTok{\# Summary}
\FunctionTok{summary}\NormalTok{(logistic\_model)}
\end{Highlighting}
\end{Shaded}

\begin{verbatim}
## 
## Call:
## glm(formula = type ~ radius, family = "binomial", data = train_reg)
## 
## Deviance Residuals: 
##     Min       1Q   Median       3Q      Max  
## -1.5427   0.0000   0.0000   0.6457   1.0839  
## 
## Coefficients:
##             Estimate Std. Error z value Pr(>|z|)   
## (Intercept)   0.1926     0.2857   0.674  0.50009   
## radius        3.7328     1.4241   2.621  0.00876 **
## ---
## Signif. codes:  0 '***' 0.001 '**' 0.01 '*' 0.05 '.' 0.1 ' ' 1
## 
## (Dispersion parameter for binomial family taken to be 1)
## 
##     Null deviance: 173.02  on 191  degrees of freedom
## Residual deviance: 117.99  on 190  degrees of freedom
## AIC: 121.99
## 
## Number of Fisher Scoring iterations: 17
\end{verbatim}

\begin{Shaded}
\begin{Highlighting}[]
\CommentTok{\# Predict test data based on model}
\NormalTok{predict\_reg }\OtherTok{\textless{}{-}} \FunctionTok{predict}\NormalTok{(logistic\_model, test\_reg, }\AttributeTok{type =} \StringTok{"response"}\NormalTok{)}
\NormalTok{predict\_reg  }
\end{Highlighting}
\end{Shaded}

\begin{verbatim}
##         2         3        13        15        22        28        32        33 
## 0.6831394 0.6395437 0.7159110 0.6715560 0.5593935 0.5598167 1.0000000 1.0000000 
##        34        47        48        51        54        55        62        76 
## 0.9999830 1.0000000 1.0000000 1.0000000 1.0000000 1.0000000 0.7113339 0.8174432 
##        80        87        93        96        99       101       105       111 
## 0.6886081 0.5565026 0.9593095 1.0000000 1.0000000 1.0000000 1.0000000 1.0000000 
##       119       133       142       144       147       152       158       165 
## 1.0000000 0.7940982 0.5583444 0.5569816 0.5567329 0.9999233 1.0000000 1.0000000 
##       169       177       182       183       184       186       188       196 
## 1.0000000 1.0000000 0.6515010 0.6599276 0.6366992 0.6354895 0.6376484 0.9366490 
##       197       200       201       210       222       228       239       240 
## 0.7605509 0.9377475 0.5591727 0.5572211 1.0000000 1.0000000 1.0000000 1.0000000
\end{verbatim}

\begin{Shaded}
\begin{Highlighting}[]
\CommentTok{\# Evaluating model accuracy}
\CommentTok{\# using confusion matrix}
\FunctionTok{table}\NormalTok{(test\_reg}\SpecialCharTok{$}\NormalTok{type, predict\_reg)}
\end{Highlighting}
\end{Shaded}

\begin{verbatim}
##                predict_reg
##                 0.556502612557871 0.556732921474065 0.556981627711086
##   Brown Dwarf                   0                 0                 0
##   Red Dwarf                     0                 0                 0
##   White Dwarf                   1                 1                 1
##   Main Sequence                 0                 0                 0
##   Supergiant                    0                 0                 0
##   Hypergiant                    0                 0                 0
##                predict_reg
##                 0.557221095610294 0.558344395704132 0.559172677832303
##   Brown Dwarf                   0                 0                 0
##   Red Dwarf                     0                 0                 0
##   White Dwarf                   1                 1                 1
##   Main Sequence                 0                 0                 0
##   Supergiant                    0                 0                 0
##   Hypergiant                    0                 0                 0
##                predict_reg
##                 0.559393497645272 0.559816669867565 0.635489485595349
##   Brown Dwarf                   0                 0                 1
##   Red Dwarf                     0                 0                 0
##   White Dwarf                   1                 1                 0
##   Main Sequence                 0                 0                 0
##   Supergiant                    0                 0                 0
##   Hypergiant                    0                 0                 0
##                predict_reg
##                 0.63669917389658 0.637648432291747 0.639543725637416
##   Brown Dwarf                  1                 1                 1
##   Red Dwarf                    0                 0                 0
##   White Dwarf                  0                 0                 0
##   Main Sequence                0                 0                 0
##   Supergiant                   0                 0                 0
##   Hypergiant                   0                 0                 0
##                predict_reg
##                 0.65150101058472 0.659927617900358 0.671555978168669
##   Brown Dwarf                  1                 1                 0
##   Red Dwarf                    0                 0                 1
##   White Dwarf                  0                 0                 0
##   Main Sequence                0                 0                 0
##   Supergiant                   0                 0                 0
##   Hypergiant                   0                 0                 0
##                predict_reg
##                 0.683139354844538 0.68860806001041 0.711333893022965
##   Brown Dwarf                   1                0                 1
##   Red Dwarf                     0                1                 0
##   White Dwarf                   0                0                 0
##   Main Sequence                 0                0                 0
##   Supergiant                    0                0                 0
##   Hypergiant                    0                0                 0
##                predict_reg
##                 0.715910958466627 0.760550877182354 0.794098228432678
##   Brown Dwarf                   0                 0                 0
##   Red Dwarf                     1                 1                 1
##   White Dwarf                   0                 0                 0
##   Main Sequence                 0                 0                 0
##   Supergiant                    0                 0                 0
##   Hypergiant                    0                 0                 0
##                predict_reg
##                 0.817443217838594 0.936649007629988 0.937747502932836
##   Brown Dwarf                   0                 0                 0
##   Red Dwarf                     1                 1                 1
##   White Dwarf                   0                 0                 0
##   Main Sequence                 0                 0                 0
##   Supergiant                    0                 0                 0
##   Hypergiant                    0                 0                 0
##                predict_reg
##                 0.959309487242502 0.999923345550524 0.999982968807287
##   Brown Dwarf                   0                 0                 0
##   Red Dwarf                     0                 0                 0
##   White Dwarf                   0                 0                 0
##   Main Sequence                 1                 1                 1
##   Supergiant                    0                 0                 0
##   Hypergiant                    0                 0                 0
##                predict_reg
##                 0.999999999949513 0.999999999991586 0.999999999996012
##   Brown Dwarf                   0                 0                 0
##   Red Dwarf                     0                 0                 0
##   White Dwarf                   0                 0                 0
##   Main Sequence                 1                 1                 1
##   Supergiant                    0                 0                 0
##   Hypergiant                    0                 0                 0
##                predict_reg
##                 0.999999999996434 0.999999999998246 1
##   Brown Dwarf                   0                 0 0
##   Red Dwarf                     0                 0 0
##   White Dwarf                   0                 0 0
##   Main Sequence                 1                 1 0
##   Supergiant                    0                 0 8
##   Hypergiant                    0                 0 8
\end{verbatim}

\begin{Shaded}
\begin{Highlighting}[]
\NormalTok{missing\_classerr }\OtherTok{\textless{}{-}} \FunctionTok{mean}\NormalTok{(predict\_reg }\SpecialCharTok{!=}\NormalTok{ test\_reg}\SpecialCharTok{$}\NormalTok{type)}
\FunctionTok{print}\NormalTok{(}\FunctionTok{paste}\NormalTok{(}\StringTok{\textquotesingle{}Accuracy =\textquotesingle{}}\NormalTok{, }\DecValTok{1} \SpecialCharTok{{-}}\NormalTok{ missing\_classerr))}
\end{Highlighting}
\end{Shaded}

\begin{verbatim}
## [1] "Accuracy = 0"
\end{verbatim}

Como puede verse, sorprendentemente el radio si que tiene cierta
relevancia a la hora de generar nuestro modelo, por lo tanto, hicimos
bien en mantener dicha variable en nuestro conjunto de datos.

\hypertarget{conclusiones}{%
\section{Conclusiones}\label{conclusiones}}

Para leer las conclusiones se ruega leer el documento de entrega anexo:
PRA2\_rmoyav.pdf

\hypertarget{exportaciuxf3n-de-datos}{%
\section{Exportación de datos}\label{exportaciuxf3n-de-datos}}

Finalmente exportaremos los datos que han sido tratados mediante la
siguiente linea de codigo.

\begin{Shaded}
\begin{Highlighting}[]
\FunctionTok{write.csv}\NormalTok{(stars, }\AttributeTok{file =} \StringTok{"../data/clean\_data.csv"}\NormalTok{)}
\end{Highlighting}
\end{Shaded}


\end{document}
